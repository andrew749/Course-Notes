\documentclass[12pt]{article}
\usepackage{amsmath}
\begin{document}
\section{GCD Theorems}
\subsection{GCD WR}
If a and b are integers not both zero, and q and r are integers such that
$a=qb+r$, then gcd(a,b)=gcd(b,r)
\subsection{GCD CT (GCD characterization theorem)}
If d is a positive common divisor of the integers a and b, and there exist
integers x and y so that $ax+by=d$,then $d=gcd(a,b)$.
\subsection{Coprimeness and Divisibility}
If a,b, and c are integers and $c\mid ab$ and $gcd(a,c)=1$, then $c\mid b$
\subsection{Primes and Divisibility}
If p is a prime and $p\mid ab$ then $p\mid a$ or $p\mid b$
\subsection{GCD of One}
Let a and b be integers. Then $gcd(a,b)=1 \iff ax+by=1$ where x and y are
integers.
\subsection{Division by the GCD}
Let a and b be integers. If $gcd(a,b)=d\neq 0$, then $gcd(\frac{a}{d},
\frac{b}{d})=1$.
\section{EEA}
\subsection{Extended Euclidean Algorithm}
If $a>b>0$ are positive integers, then $d=gcd(a,b)$ can be computed and there
exist integers x and y so that $ax+by=d$.
\section{Linear Diophatine Equations}
Equations with integer co-efficients for which integer solutions are sought.
\subsection{Linear Diophatine Equation Theorem, Part 1 LDET1}
Let gcd(a,b)=d. The linear Diophatine equation $ax+by=c$ has a solution iff
$d\mid c$
\subsection{Linear Diophatine Equation Theorem 2, LDET2}
Let $gcd(a,b)=d$ where both a and b are not zero. If $x=x_0$ and $y=y_0$ is one
particular integer solution to the equation $ax+by=d$ then the complete
solution is $x=x_0+\frac{b}{d} n$, $y=y_0-\frac{a}{d} n \forall \epsilon Z$
\section{Congruence}
Let m be a fixed positive integer. If $a,b \epsilon Z$ we say that a is
congruent to b modulo m. $a\equiv b mod m$ if $m\mid (a-b)$. If $m \nmid (a-b)$
then $a\neq mod m$
\subsection{Equivalence relation}
\begin{enum}
\item $a\equiv b mod m$
\item If $a\equiv b mod m $ then $b\equiv a mod m$
\item If $a\equiv b mod m$ and $b\equiv c mod m$ then $a\equiv c mod m$
\end{enum}
\subsection{properties of congruence}
\begin{enum}
\item $a+b\equiv a'+b' mod m$
\item $a-b\equiv a'-b'mod m$
\item $ab\equiv a'b'mod m$
\end {enum}
\subsection{Congruences and Division}
If $ac\equiv bc mod m$ and $gcd(c,m)=1$ then $a\equiv b mod m$
\subsection{Congruent IFF Same remainder}
$a\equiv b mod m$ iff a and b have the same remainder when divided by m.
\subsection{Modular Arithmetic}
\subsubsection{Congruence class}
$[a]={x\epsilon Z\mid x\equiv a mod m}$
\\
$Z_m$ is the set of m congruence classes. $[a]+[b]=[a+b]$ and $[a]*[b]=[a*b]$
\subsubsection{Identity}
Something that does nothing. \\ $\forall a \epsilon S, a * e=a$
\subsubsection{Inverse}
$a*b=b*a=e$
Subtraction is the addition of inverse.
Division is multiplication with inverse.
\subsection{Fermat's Little Theorem}
If p is a prime number that does not divide the integer a, then $a^{p-1}
\equiv 1 mod p$
For any integer a and any prime p $a^p\equiv a mod p$
\subsubsection{Existence of Inverse}
Let p be a prime. if $[a]$ is any non zero element in $Z_p$ then there exists an
element $[b]\epsilon Z_p$ so that $[a]*[b]=1$
\subsection{Linear Congruences}

\end{document}

