\documentclass[12pt]{article}
\usepackage{amsmath}
\usepackage{amssymb}
\usepackage{amsfonts}
\usepackage{graphicx}
\begin{document}
\section{GCD Theorems}
\subsection{GCD WR}
If a and b are integers not both zero, and q and r are integers such that
$a=qb+r$, then $\gcd(a,b)=\gcd(b,r)$
\subsection{GCD CT (GCD characterization theorem)}
If d is a positive common divisor of the integers a and b, and there exist
integers x and y so that $ax+by=d$,then $d=\gcd(a,b)$.
\subsection{Coprimeness and Divisibility}
If a,b, and c are integers and $c\mid ab$ and $\gcd(a,c)=1$, then $c\mid b$
\subsection{Primes and Divisibility}
If p is a prime and $p\mid ab$ then $p\mid a$ or $p\mid b$
\subsection{GCD of One}
Let a and b be integers. Then $\gcd(a,b)=1 \iff ax+by=1$ where x and y are
integers.
\subsection{Division by the GCD}
Let a and b be integers. If $\gcd(a,b)=d\neq 0$, then $\gcd(\frac{a}{d},
\frac{b}{d})=1$.
\section{EEA}
\subsection{Extended Euclidean Algorithm}
If $a>b>0$ are positive integers, then $d=\gcd(a,b)$ can be computed and there
exist integers x and y so that $ax+by=d$.
\section{Linear Diophatine Equations}
Equations with integer co-efficients for which integer solutions are sought.
\subsection{Linear Diophatine Equation Theorem, Part 1 LDET1}
Let gcd (a,b)=d. The linear Diophatine equation $ax+by=c$ has a solution iff
$d\mid c$
\subsection{Linear Diophatine Equation Theorem 2, LDET2}
Let $\gcd(a,b)=d$ where both a and b are not zero. If $x=x_0$ and $y=y_0$ is one
particular integer solution to the equation $ax+by=d$ then the complete
solution is $x=x_0+\frac{b}{d} n$, $y=y_0-\frac{a}{d} n \forall \epsilon Z$
\section{Congruence}
Let m be a fixed positive integer. If $a,b \epsilon Z$ we say that a is
congruent to b modulo m. $a\equiv b \mod{m}$ if $m\mid (a-b)$. If $m \nmid (a-b)$
then $a\neq \mod{m}$
\subsection{Equivalence relation}
\begin{enumerate}
  \item{$a\equiv b \mod{m}$}
  \item{If $a\equiv b \mod{m} $ then $b\equiv a \mod{m}$}
  \item{If $a\equiv b \mod{m}$ and $b\equiv c \mod{m}$ then $a\equiv c \mod{m}$}
\end{enumerate}
\subsection{properties of congruence}
\begin{enumerate}
  \item{$a+b\equiv a'+b' \mod{m}$}
  \item{$a-b\equiv a'-b'\mod{m}$}
  \item{$ab\equiv a'b'\mod{m}$}
\end{enumerate}
\subsection{Congruences and Division}
If $ac\equiv bc \mod{m}$ and $\gcd(c,m)=1$ then $a\equiv b \mod{m}$
\subsection{Congruent IFF Same remainder}
$a\equiv b \mod(m)$ iff a and b have the same remainder when divided by m.
\subsection{modular Arithmetic}
\subsubsection{Congruence class}
$[a]={x\epsilon Z\mid x\equiv a \mod(m)}$
\\
$Z_m$ is the set of m congruence classes. $[a]+[b]=[a+b]$ and $[a]*[b]=[a*b]$
\subsubsection{Identity}
Something that does nothing. \\ $\forall a \epsilon S, a * e=a$
\subsubsection{Inverse}
$a*b=b*a=e$
Subtraction is the addition of inverse.
Division is multiplication with inverse.
\subsection{Fermat's Little Theorem}
If p is a prime number that does not divide the integer a, then $a^{p-1}
\equiv 1 \mod(p)$
For any integer a and any prime p $a^p\equiv a \mod p$
\subsubsection{Existence of Inverse}
Let p be a prime.\ if $[a]$ is any non zero element in $Z_p$ then there exists an
element $[b]\epsilon Z_p$ so that $[a]*[b]=1$
\subsection{Linear Congruences}
$ax\equiv c mod m$ is a linear congruence.\ solution if x so that congruence is
true.
\subsubsection{Linear congruence Theorem 1 LCT1}
Let $gcd(a,m)=d\neq 0$ The linear congruence $ax\equiv c mod m$ has a solution
iff $d\mid c$. Also if $x_0$ is a solution then complete solution is $x\equiv x_0
mod \frac{m}{d}$
\subsubsection{Linear Congruence Theorem 2, LCT2}
Let $gcd(a,m)=d\neq 0$. The equation $[a][c]=[c]$ in $Z_m$ has a solution iff
$d\mid c$
\subsection{Chinese Remainder Theorem}
Let $a_1, a_2 \epsilon Z$ If $gcd(m_1, m_2)=1$ then the simultaneous linear
congruences $n\equiv a_1mod m_1$ and $n\equiv a_2 \mod m_2 $ have a unique
solution \modulo $m_1m_2$ Thus is n=$n_0$ is one integer solution then the
complete solution is $n\equiv n_0 mod m_1 m_2$
\section{RSA}
\subsection{Setting up RSA}
\begin{enumerate}

  \item choose two large distinct primes p and q and let n=pq.
  \item select an integer e so that $gcd(e,(p-1)(q-1))=1$ and $1<e<(p-1)(q-1)$
  \item solve $ed\equiv 1(mod (p-1)(q-1))$for an integer $1<d<(p-1)(q-1)$
  \item public key is (e,m) and private key is (d,n)
\end{enumerate}
\subsection{sending a message}
\begin{enumerate}
  \item look up public key(e,n)
  \item generate an integer message M $0\leq M < n$
  \item compute the ciphertext C $M^e\equiv C mod n$ where $0\leq C < n$
\end{enumerate}
\subsection{receiving a message}
\begin{enumerate}
  \item use the private key (d,n)
  \item compute message text R from C $C^d\equiv R (mod n)$ where $0\leq R <n$
\end{enumerate}
\section{Complex Numbers}
\textbf{Complex Numbers:}An expression of the form \  $C= {x+yi \mid x,y \epsilon R}$
where x and y are real numbers. Has a real and imaginary part.
\begin{enumerate}{Mathematical operations}
  \item{$a+bi=c+di \leftrightarrow a=c & b=d$}
  \item{$(a+bi)+(c+di)=(a+c)+(b+d)i$}
  \item{$(a+bi)(c+di)=(ac-bd)+(ad+bc)i$}
\end{enumerate}
\begin{enumerate}
  \item Associativity of addition $(u+v)+z=u+(v+z)$
  \item Commutativity of addition $(u+v)=v+u$
  \item additive identity $0+0i$
  \item Additive inverse $-z=-x-yi$
  \item Associativity of multiplication $(u*v)*z=u*(v*z)$
  \item Commutativity of multiplication $(u*v)=v*u$
  \item multiplicative identity $1=1+0i$
  \item multiplicative inverse $z=x+yi \frac{x-yi}{x^2+y^2}$
\end{enumerate}
\subsection{Complex conjugate}
complex conjugate of $z=x+yi$ is \ $ \overline{z}=x-y$
\subsection{modulus}
modulus of $z=x+yi$ $|z|=|x+yi|=\sqrt{x^2+y^2}$
\subsection{Polar multiplication of Complex Numbers}
If $z_1=r_1(cos\theta + isin\theta)$ and $z_2=r_2(cos\theta +isin\theta)$ are
two complex numbers in polar form then
$z_1z_2=r_1r_2(cos(\theta_1+\theta_2)+isin(\theta_1+\theta_2))$
\subsection{De Moivre's Theorem}
If $\theta \epsilon \mathbb{R}$ and $n\epsilon  \mathbb{Z}$ then
$(cos\theta+isin\theta)^n=cos(n\theta)+ isin(n\theta)$\\
if $z=r(cos\theta +isin\theta)$ and n is an integer then $z^n=r^n(cos\theta
+isin\theta)^n$
\subsection{complex exponential function}
$e^{i*\theta}= cos\theta+isin\theta$
\subsection{Properties of Complex Exponentials}
$e^{i\theta}*e^{i\phi}=e^{i(\theta+\phi)}$\\
$(e^{i\theta})^n=e^{in\theta}$
\subsection{Complex n-th Roots Theorem}
if $r(cos\theta + isin\theta)$ is in the polar form of a complex number a, then
the solutions are $\sqrt[n]{r}(cos(\frac{\theta + 2\pi
  k}{n})+isin(\frac{\theta+2\pi k}{n}))$
\section{Polynomials}
\subsection{Definition}
A polynomial in x over the field \mathbb{F} is an expression of the form
$a_n*x^n + a_{n+1}x^{n-1} + \cdots +a_1x+a_0$ all of the a's are
coefficients.\\
If $a_n\neq0$ in $a_n*x^n + a_{n+1}x^{n-1} + \cdots +a_1x+a_0$ Then  polynomial
has degree n. Zero has all coefficients 0 and degree undefined.
\subsection{Equality}
only equal if $a_i=b_i$ for all a,b
\subsection{Sum}
sum of f(x) and g(x) is $f(x)+g(x)=\sum_{i=0}^{max(n,m)}(a_i+b_i)x^i$ with any
missing terms coefficient 0.
\subsection{Difference}
difference of f(x) and g(x) is $f(x)-g(x)=\sum_{i=0}^{max(n,m)}(a_i-b_i)x^i$ with any
missing terms coefficient 0.
\subsection{Products}
product of f(x) and g(x) is $f(x)*g(x)=\sum_{i=0}^{n+m}c_ix^i$where
$c_i=\sum_{j=0}^{i} a_jb_{i-j}$
\subsection{Division algorithm for polynomials DAP}
If f(x) and g(x) are polynomials in \mathbb{F}[x] and g(x) is not the zero
polynomial, then there exist unique polynomials q(x) and r(x) in \mathbb{F}
such that $f(x)=q(x)g(x) +r(x)$ where $deg(r(x))<deg(g(x))$
\includegraphics{p95.gif}
\end{document}

