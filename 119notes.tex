\documentclass{article}
\usepackage{amsmath}
\usepackage{mathtools}
\usepackage{amsthm}

\begin{document}

\title{MATH119}

\section{Approximation Methods}
\subsection{Linear Approximation}
Can form a tangent line of f(x) at a with the derivative f'(x)
\begin{equation}
L(x)=f(a)+f'(a)(x-a)
\end{equation}
\subsubsection{Solution Methods}
\paragraph{Bisection Method}
\begin{enumerate}
\item Determine interval (a,b) on which function has a root.
\item Divide interval into sub intervals: $(a,\frac{a_b}{2}),(\frac{a+b}{2},b)$
\item Determine which interval solution lives in.
\item Apply 2 again with smaller interval
\end{enumerate}
Repeat until reach desired accuracy.
\paragraph{Newtons method}
Rearrange linearization to $x_(n+1)=x_n-\frac{f(x_n)}{f'(x_n)}$
\paragraph{Fixed Point Iteration}
Let f(x) be differentiable and suppose f(x) has a solution. If |f'(x)|<1 for all values of x within an interval of the fixed point then FPI converges for any x.

Basically rearrange equation for x and use result of one solution as argument for next.

If guess is not close though function WILL diverge.
\subsection{Polynomial Interpolation}
2 points can determine a line in $R^2$
3 points determine a quadratic in $R^2$
\subsubsection{Newton Forward Difference Formula}
\begin{equation}
y=y_0+(x-x_0)\frac{\triangle y_0}{\triangle x}+ \dots \frac{x(x-x_0)(x-x_1)...(x-(n-1))\triangle^n y_0}{h^n n!}
\end{equation}
Where h is the difference in space between elements and $x_n=x_0+nh$
\subsubsection{Taylor Polynomial}
\begin{equation}
P_(k,x_0)(x)=\sum\limits_{n=0}^k \frac{f^n x_0(x-x_0)^n}{n!}
\end{equation} 
Far away from $x_0$ the approximations can get bad.
\begin{equation}
a_k=\frac{1}{k!} f^{(k)}(x_0)
\end{equation}
\paragraph{Example $f(x)=e^{x^2}$}
Find for $P_{(4,0)}(x)$ for $e^t$\newline
\begin{align}
P_{(4,0)}(t)=1+t+1/2t^2+1/6t^3+a/24t^4\\
P_{(8,0)}(t)=1+x^2+1/2x^4+1/6x^6+a/24x^8
\end{align}

Mclaurin TP with $x_0=0$



Fundamental theorem of Calculus: if F'(x)=f(x) 
then\\
\begin{equation}
\int_a^b f(t) \mathrm{d}t=F(b)-F(a)
\end{equation}
Taylor Theorem\\
$f(x)=P{(n,x_0)}(x)+\int_{x_0}^x \frac{(x-t)^n}{n!} f^{n+1}(t) \mathrm{d}t$

We know that the remainder must be equal to the actual function minus the taylor polynomial by definition. 
Accept that $|f^{n+1}(t)|<=k$ for all $t \epsilon [x_0,x]$\\
Error is equal to :

$|f(x)-P_{(n,x_0)}(x)| \leq | \int_{x_0}^x \frac{(x-t)^n}{n!} f^{n+1}(t) \mathrm{d}t| \leq \frac{k}{(n+1)!} |x-x_0|^{n+1}$
\subsubsection{Infinite Series}

\end{document}